\documentclass[11pt,a4paper, uplatex]{jsarticle}
%
\usepackage{amsmath,amssymb}
\usepackage{bm}
\usepackage[dvipdfmx]{graphicx}
\usepackage{ascmac}
\usepackage{listings}
\usepackage{underscore}
\lstset{
    frame=single,
    numbers=left,
    tabsize=2
}
%
\setlength{\textwidth}{\fullwidth}
\setlength{\textheight}{40\baselineskip}
\addtolength{\textheight}{\topskip}
\setlength{\voffset}{-0.2in}
\setlength{\topmargin}{0pt}
\setlength{\headheight}{0pt}
\setlength{\headsep}{0pt}
%
\newcommand{\divergence}{\mathrm{div}\,}  %ダイバージェンス
\newcommand{\grad}{\mathrm{grad}\,}  %グラディエント
\newcommand{\rot}{\mathrm{rot}\,}  %ローテーション
%
\title{メディア情報学実験・メディア分析 課題レポート}
\author{1510151  栁 裕太}
\date{\today}
\begin{document}
\maketitle
\section{序論・仮説}
今回の実験では、"きよしのズンドコ節"という曲を扱った(PV16)。
このPVでは昭和の歌謡曲のエッセンスを入れたことによって、
筆者のような20代前半にとっては個人的にはかなり印象深いPVであった。
よって今回は、以下の仮説を立ててから解析に臨むことにした。

\begin{itemize}
  \item PVを構成する任意の要素が洗練されていなくとも、好感度には影響しない
  \item 映像・メロディに迫力がなくとも、好感度には影響しない
  \item 聞き取りやすいメロディ・歌詞は好感度上昇に寄与する
\end{itemize}

\section{調査結果分析}
\subsection{主成分抜粋}
主成分抜粋においては、累積寄与率と固有値の2つのデータを基準に足切りを行った。
なお、PC11以降は省略している。

\begin{table}[htbp]
  \begin{center}
    \begin{tabular}{c|c|c}
      \hline
      主成分番号 & 累積寄与率(\%) & 固有値 \\ \hline \hline
      PC1 & 38.38390791 & 7.29294250348674 \\
      PC2 & 48.55050945 & 1.93165429251404 \\
      PC3 & 55.79636735 & 1.37671299972187 \\
      PC4 & 61.89941496 & 1.15957904665 \\
      PC5 & 67.01570809 & 0.972095695622575 \\
      PC6 & 71.71468219 & 0.892805078337444 \\
      PC7 & 75.76539908 & 0.769636208492113 \\
      PC8 & 79.51202683 & 0.711859273699518 \\
      PC9 & 82.66178688 & 0.598454408935556 \\
      PC10 & 85.20054444 & 0.482363935634707 \\
      \hline
    \end{tabular}
  \end{center}
\end{table}

講義内では、以下の条件で足切りすることが推奨されていた。

\begin{itemize}
  \item 累積寄与率が80\%以下の主成分
  \item 固有値が1以上の主成分
\end{itemize}

前者であればPC8、後者であればPC4までとなるが、
両者のデータ共に値が著しく変化する境界があまり明瞭ではない。
そこで、前者の広い基準を採用し、解析後のP値等によって解析対象から外すことにした。

\subsection{目的関数に寄与する主成分の選定・再解析}
\subsection{主成分を構成する質問・主成分命名}
\subsection{グループの類推}


\section{結論}
\section{考察}



\end{document}
