\documentclass[11pt,a4paper, uplatex]{jsarticle}
%
\usepackage{amsmath,amssymb}
\usepackage{bm}
\usepackage[dvipdfmx]{graphicx}
\usepackage{ascmac}
\usepackage{listings}
\usepackage{underscore}
\usepackage{subfig}
\lstset{
    frame=single,
    numbers=left,
    tabsize=2
}
%
\setlength{\textwidth}{\fullwidth}
\setlength{\textheight}{40\baselineskip}
\addtolength{\textheight}{\topskip}
\setlength{\voffset}{-0.2in}
\setlength{\topmargin}{0pt}
\setlength{\headheight}{0pt}
\setlength{\headsep}{0pt}
%
\newcommand{\divergence}{\mathrm{div}\,}  %ダイバージェンス
\newcommand{\grad}{\mathrm{grad}\,}  %グラディエント
\newcommand{\rot}{\mathrm{rot}\,}  %ローテーション
%
\title{メディア情報学実験・メディア分析 課題レポート}
\author{1510151  栁 裕太}
\date{\today}
\begin{document}
\maketitle
\section{序論・仮説}
\label{sc:intro}
今回の実験では、"きよしのズンドコ節"という曲のPVを扱った(PV16)。
このPVでは昭和の歌謡曲のエッセンスを入れたことによって、
筆者のような20代前半にとっては個人的にはかなり印象深いPVであった。
よって今回は、以下の仮説を立ててから解析に臨むことにした。

\begin{itemize}
  \item PVを構成する任意の要素が洗練されていなくとも、好感度には影響しない
  \item 映像・メロディに迫力がなくとも、好感度には影響しない
  \item 聞き取りやすいメロディ・歌詞は好感度上昇に寄与する
\end{itemize}

\section{調査結果分析}
\subsection{a-sum分析(調査全体)}
\subsubsection{主成分抜粋}
主成分抜粋においては、累積寄与率と固有値の2つのデータを基準に足切りを行った。
なお、PC5以降は省略している。

\begin{table}[htbp]
  \begin{center}
    \caption{全体調査に対する主成分毎の累積寄与率と固有値}
    \label{tb:allpcselect}
    \begin{tabular}{c|c|c}
      \hline
      主成分番号 & 累積寄与率(\%) & 固有値 \\ \hline \hline
      PC1 & 43.83105671 & 8.327900774 \\
      PC2 & 68.24472267 & 4.638596532 \\
      PC3 & 83.05686358 & 2.814306774 \\
      PC4 & 90.02127311 & 1.32323781 \\
      PC5 & 93.34194823 & 0.630928273 \\
      \hline
    \end{tabular}
  \end{center}
\end{table}

講義内では、以下の条件で足切りすることが推奨されていた。

\begin{itemize}
  \item 累積寄与率が80\%以下の主成分
  \item 固有値が1以上の主成分
\end{itemize}

前者であればPC2、後者であればPC4までとなるが、
PC2で足切りした場合、固有値が2を上回るPC3をも切り捨てる必要が生じることになる。
よって、後者の広い基準を採用し、PC4まで後の解析にかけることにした。

\subsubsection{重回帰式による検証}
PC1からPC4まで全ての主成分を対象に重回帰分析を行った。その結果は以下の通りである。

\begin{table}[htbp]
  \begin{center}
    \caption{重回帰分析結果: 各主成分の係数}
    \begin{tabular}{c|c|c|c|c|c|c}
      \hline
      主成分番号 & 偏回帰係数 & 標準誤差 & t値 & P値 & 標準化偏回帰係数 & トレランス \\ \hline \hline
      PC1 & -25.7 & 1.07 & -23.9 & 2.29e-13 & -0.941 & 1 \\
      PC2 & 6.67 & 1.44 & 4.65 & 0.000317 & 0.183 & 1 \\
      PC3 & -10.5 & 1.84 & -5.71 & 4.11e-05 & -0.224 & 1 \\
      PC4 & 6.00 & 2.69 & 2.23 & 0.0413 & 0.0877 & 1 \\
      定数項 & 829 & 3.09 & 268 & 5.06e-29 & NA & NA \\
    \end{tabular}
    \label{tb:allone}
  \end{center}
\end{table}

\begin{table}[htbp]
  \begin{center}
    \caption{再重回帰分析結果: 重相関係数・自由度調整済重相関係数}
    \begin{tabular}{c|c|c}
      \hline
      重相関係数 & 重相関係数の2乗 & 自由度調整済重相関係数の2乗 \\ \hline \hline
      0.988353434669365 & 0.97684251182273 & 0.970667181642125
    \end{tabular}
    \label{Rs}
  \end{center}
\end{table}

\begin{table}[htbp]
  \begin{center}
    \caption{再重回帰分析結果: 分散分析}
    \begin{tabular}{c|c|c|c|c|c}
      \hline
      項目 & 平方和 & 自由度 & 平均平方 & F値 & P値 \\ \hline \hline
      回帰 & 1.21e+05 & 4 & 3.02e+04 & 158 & 4.52e-12 \\
      残差 & 2.87e+03 & 15 & 191 & NA & NA \\
      全体 & 1.24e+05 & 19 & 6.51e+03 & NA & NA \\
    \end{tabular}
    \label{annova}
  \end{center}
\end{table}

これを重回帰式にすると、以下の通りとなった。なお、目的変数は$ PV_{like} $とした。

\begin{equation}
  \begin{split}
    PV_{like} &= -25.7PC_1 + 6.67PC_2 - 10.5PC_3 + 6.00PC_4 + 829
  \end{split}
\end{equation}

この目的変数$ PV_{like} $に対して$ PC_1 $から$ PC_4 $までを説明変数として重回帰分析を
行った結果、決定係数は$ R^2 = .98 $であり、0.1\%水準で有意であった。
($ F(4, 15) = 158 $)

\subsubsection{目的関数に寄与する主成分の選定}
\label{sc:pval}

主成分の選定において着目したのがP値の列である。
講義内では$ p < 0.005 $が望ましいとされた。
今回の解析では、表\ref{tb:allone}のP値の列によると、
いずれの主成分もこの基準を満たしているため、
全ての主成分を残すべきであることが適当であると判断した。

\subsubsection{主成分を構成する質問・主成分命名}
\label{sc:allfaclor}
PV16の主成分を構成する質問の選出には、各質問に対応した主成分負荷量を対象とし、
それらの絶対値が大きい値を持つ質問を選抜することにした。

以下の表\label{tb:allpcs}が、
各主成分に対して主成分負荷量の絶対値が最も大きかった5つの質問とその負荷量を表している。

\begin{table}[h]
  \begin{center}
    \caption{各主成分に対する主成分負荷量絶対値TOP5}
    \label{tb:allpcs}
    \subfloat[PC1]{
      \begin{tabular}{c|c}
        \hline
        質問番号 & 主成分負荷量 \\ \hline \hline
        Q14 & -0.935 \\
        Q17 & -0.933 \\
        Q19 & -0.902 \\
        Q3 & -0.876  \\
        Q11 & -0.857 \\
        \hline
      \end{tabular}
    }
    \subfloat[PC2]{
      \begin{tabular}{c|c}
        \hline
        質問番号 & 主成分負荷量 \\ \hline \hline
        Q7 & 0.874 \\
        Q1 & 0.784 \\
        Q9 & 0.781 \\
        Q10 & 0.703 \\
        Q13 & 0.659 \\
        \hline
      \end{tabular}
    }
    \vspace{1cm}
    \subfloat[PC3]{
      \begin{tabular}{c|c}
        \hline
        質問番号 & 主成分負荷量 \\ \hline \hline
        Q1 & 0.566 \\
        Q4 & 0.533 \\
        Q18 & -0.499 \\
        Q9 & -0.495 \\
        Q15 & 0.460 \\
        \hline
      \end{tabular}
    }
    \subfloat[PC4]{
      \begin{tabular}{c|c}
        \hline
        質問番号 & 主成分負荷量 \\ \hline \hline
        Q8 & -0.615 \\
        Q2 & -0.401 \\
        Q18 & -0.398 \\
        Q14 & 0.296 \\
        Q6 & 0.281 \\
        \hline
      \end{tabular}
    }
  \end{center}
\end{table}

このデータを元に、各主成分に命名した一覧は以下の通りである。

\begin{description}
  \item[PC1] {\bf 好感度・購買欲}
  \begin{itemize}
    \item 各要素に対する好感度を問うた質問が主成分負荷量絶対値の上位の多くを占めたため
    \item それに加え、Q19:購買意欲を問うた質問も上位に入っていたため
  \end{itemize}

  \item[PC2] {\bf 映像・雰囲気のクオリティ}
  \begin{itemize}
    \item 映像の洗練性と迫力、そして雰囲気の問うた質問が主成分負荷量絶対値の上位3つであったため
    \item その2問の主成分負荷量絶対値が以下の質問に対して差をあけていたため
  \end{itemize}

  \item[PC3] {\bf 映像・アーティストの洗練性}
  \begin{itemize}
    \item 映像・アーティストの洗練性を問うた質問が主成分負荷量絶対値の上位2つを占めたため
  \end{itemize}

  \item[PC4] {\bf 映像の見やすさ}
  \begin{itemize}
    \item 映像の見やすさを問うたQ8が主成分負荷量絶対値の最上位だったため
    \item Q8以降との主成分負荷量絶対値の差が0.2以上開いていたため
  \end{itemize}
\end{description}

\subsubsection{グループの類推}
調査結果に対して、クラスタ解析を行った結果が以下の図\ref{fig:clusterALL}の通りである。

\begin{figure}[h]
  \begin{center}
    \includegraphics[width=10cm]{clusterALL.bmp}
    \caption{全体調査結果に対するクラスタ分析結果}
    \label{fig:clusterALL}
  \end{center}
\end{figure}
目的変数Q20に最も近い位置の質問がQ17となった。
そのQ14は、表\ref{tb:allpcs}よりPC1を構成する質問の1つでもあった。
また、図左部分に各要素の洗練性を問うた質問が集中している他、
各主成分を構成していた質問が近い関係にある場合が多かった。

\subsection{PV16分析(きよしのズンドコ節)}
\subsubsection{主成分抜粋}
主成分抜粋においては、累積寄与率と固有値の2つのデータを基準に足切りを行った。
なお、PC11以降は省略している。

\begin{table}[htbp]
  \begin{center}
    \caption{PV16に対する主成分毎の累積寄与率と固有値}
    \label{tb:pv16pcs}
    \begin{tabular}{c|c|c}
      \hline
      主成分番号 & 累積寄与率(\%) & 固有値 \\ \hline \hline
      PC1 & 38.38390791 & 7.29294250348674 \\
      PC2 & 48.55050945 & 1.93165429251404 \\
      PC3 & 55.79636735 & 1.37671299972187 \\
      PC4 & 61.89941496 & 1.15957904665 \\
      PC5 & 67.01570809 & 0.972095695622575 \\
      PC6 & 71.71468219 & 0.892805078337444 \\
      PC7 & 75.76539908 & 0.769636208492113 \\
      PC8 & 79.51202683 & 0.711859273699518 \\
      PC9 & 82.66178688 & 0.598454408935556 \\
      PC10 & 85.20054444 & 0.482363935634707 \\
      \hline
    \end{tabular}
  \end{center}
\end{table}

こちらの解析でも、講義内で推奨された基準を適用することにした。

前者であればPC8、後者であればPC4までとなるが、
両者のデータ共に値が著しく変化する境界があまり明瞭ではない。
そこで、前者の広い基準を採用し、解析後のP値等によって解析対象から外すことにした。

\subsubsection{重回帰式による検証}
PC1からPC8まで全ての主成分を対象に重回帰分析を行った。その結果は以下の通りである。

\begin{table}[htbp]
  \begin{center}
    \caption{重回帰分析結果: 各主成分の係数}
    \begin{tabular}{c|c|c|c|c|c|c}
      \hline
      主成分番号 & 偏回帰係数 & 標準誤差 & t値 & P値 & 標準化偏回帰係数 & トレランス \\ \hline \hline
      PC1 & -0.282 & 0.0119 & -23.6 & 2.09e-64 & -0.765 & 1 \\
      PC2 & -0.171 & 0.0232 & -7.38 & 2.66e-12 & -0.239 & 1 \\
      PC3 & -0.197 & 0.0275 & -7.15 & 1.05e-11 & -0.232 & 1 \\
      PC4 & -0.0489 & 0.0300 & -1.63 & 0.104 & -0.0529 & 1 \\
      PC5 & 0.143 & 0.0327 & 4.39 & 1.72e-05 & 0.142 & 1 \\
      PC6 & 0.105 & 0.0341 & 3.07 & 0.00242 & 0.0994 & 1 \\
      PC7 & -0.156 & 0.0368 & -4.24 & 3.26e-05 & -0.137 & 1 \\
      PC8 & -0.00994 & 0.0382 & -0.260 & 0.795 & -0.00843 & 1 \\
      定数項 & 3.15 & 0.0323 & 97.7 & 2.64e-195 & NA & NA \\
    \end{tabular}
    \label{one}
  \end{center}
\end{table}

\begin{table}[htbp]
  \begin{center}
    \caption{再重回帰分析結果: 重相関係数・自由度調整済重相関係数}
    \begin{tabular}{c|c|c}
      \hline
      重相関係数 & 重相関係数の2乗 & 自由度調整済重相関係数の2乗 \\ \hline \hline
      0.864590803140089 & 0.747517256874424 & 0.739101165436905
    \end{tabular}
    \label{Rs}
  \end{center}
\end{table}

\begin{table}[htbp]
  \begin{center}
    \caption{再重回帰分析結果: 分散分析}
    \begin{tabular}{c|c|c|c|c|c}
      \hline
      項目 & 平方和 & 自由度 & 平均平方 & F値 & P値 \\ \hline \hline
      回帰 & 184 & 8 & 23.0 & 88.8 & 2.36e-67 \\
      残差 & 62.2 & 240 & 0.259 & NA & NA \\
      全体 & 246 & 248 & 0.993 & NA & NA \\
    \end{tabular}
    \label{annova}
  \end{center}
\end{table}

これを重回帰式にすると、以下の通りとなった。なお、目的変数は$ PV_{like} $とした。

\begin{equation}
  \begin{split}
    PV_{like} &= -0.282PC_1 - 0.171PC_2 - 0.197PC_3 - 0.0489PC_4 \\
    &\quad + 0.143PC_5 + 0.105PC_6 - 0.156PC_7 - 0.00994PC_8 + 3.15
  \end{split}
\end{equation}

この目的変数$ PV_{like} $に対して$ PC_1 $から$ PC_8 $までを説明変数として重回帰分析を
行った結果、決定係数は$ R^2 = .75 $であり、0.1\%水準で有意であった。
($ F(8, 240) = 88.8 $)

\subsubsection{目的関数に寄与する主成分の選定}
\label{onesec}

第\ref{sc:pval}節同様に、$ p < 0.005 $を基準に主成分を選定することにした。
今回の解析では、表\ref{one}のP値の列によると、
PC4, PC8がこの基準を満たしていないため、
除外して重回帰分析を再度行うことが適当であると判断した。

\subsubsection{重回帰式による再検証}
第\ref{onesec}節より、PC4, PC8を除外した状態でもう一度重回帰分析を行った。
その結果得られたデータは次の表\ref{two}の通りである。

\begin{table}[htbp]
  \begin{center}
    \caption{再重回帰分析結果: 各種成分の係数}
    \begin{tabular}{c|c|c|c|c|c|c}
      \hline
      主成分番号 & 偏回帰係数 & 標準誤差 & t値 & P値 & 標準化偏回帰係数 & トレランス \\ \hline \hline
      PC1 & -0.282 & 0.0120 & -23.5 & 1.64E-64 & -0.765 & 1 \\
      PC2 & -0.171 & 0.0232 & -7.36 & 2.79E-12 & -0.240 & 1 \\
      PC3 & -0.197 & 0.0275 & -7.14 & 1.10E-11 & -0.232 & 1 \\
      PC5 & 0.143 & 0.0328 & 4.38 & 1.77E-05 & 0.142 & 1 \\
      PC6 & 0.105 & 0.0342 & 3.06 & 0.00245 & 0.0994 & 1 \\
      PC7 & -0.156 & 0.0368 & -4.23 & 3.33E-05 & -0.137 & 1 \\
      定数項 & 3.15 & 0.0323 & 97.6 & 2.45E-196 & NA & NA \\
    \end{tabular}
    \label{two}
  \end{center}
\end{table}

\begin{table}[htbp]
  \begin{center}
    \caption{再重回帰分析結果: 重相関係数・自由度調整済重相関係数}
    \begin{tabular}{c|c|c}
      \hline
      重相関係数 & 重相関係数の2乗 & 自由度調整済重相関係数の2乗 \\ \hline \hline
      0.862928027094451 & 0.744644779945121 & 0.738313658786736
    \end{tabular}
    \label{Rs-new}
  \end{center}
\end{table}

\begin{table}[htbp]
  \begin{center}
    \caption{再重回帰分析結果: 分散分析}
    \begin{tabular}{c|c|c|c|c|c}
      \hline
      項目 & 平方和 & 自由度 & 平均平方 & F値 & P値 \\ \hline \hline
      回帰 & 183 & 6 & 30.6 & 118 & 7.69e-69 \\
      残差 & 62.9 & 242 & 0.260 & NA & NA \\
      全体 & 246 & 248 & 0.993 & NA & NA \\
    \end{tabular}
    \label{annova-new}
  \end{center}
\end{table}

これを重回帰式にすると、以下の式\ref{formula:two}の通りとなった。

\begin{equation}
  \label{formula:two}
  \begin{split}
    PV_{like} &= -0.282PC_1 - 0.171PC_2 - 0.197PC_3 - \\
    &\quad + 0.143PC_5 + 0.105PC_6 - 0.156PC_7 + 3.15
  \end{split}
\end{equation}

この分析結果をまとめると、目的変数$ PV_{like} $に対して
$ PC_1, PC_2, PC_3, PC_5, PC_6, PC_7$
までを説明変数として重回帰分析を行った結果、
決定係数は$ R^2 = .74 $であり、0.1\%水準で有意であった。
($ F(6, 242) = 118 $)

\subsubsection{主成分を構成する質問・主成分命名}
\label{sc:pvpcq}
第\ref{sc:allfaclor}節と同様に、PV16の主成分を構成する質問の選出には、
各質問に対応した主成分負荷量を対象とし、
それらの絶対値が大きい値を持つ質問を選抜することにした。

以下の表\ref{tb:PVPCs}が、各主成分に対して主成分負荷量の絶対値が最も大きかった5つの質問とその負荷量を表している。

\begin{table}[htbp]
  \begin{center}
    \caption{各主成分に対する主成分負荷量絶対値TOP5}
    \label{tb:PVPCs}
    \subfloat[PC1]{
      \begin{tabular}{c|c}
        \hline
        質問番号 & 主成分負荷量 \\ \hline \hline
        Q14 & -0.801 \\
        Q17 & -0.756 \\
        Q3 & -0.727 \\
        Q10 & -0.724  \\
        Q6 & -0.721 \\
        \hline
      \end{tabular}
    }
    \subfloat[PC2]{
      \begin{tabular}{c|c}
        \hline
        質問番号 & 主成分負荷量 \\ \hline \hline
        Q1 & 0.604 \\
        Q7 & 0.548 \\
        Q2 & -0.418 \\
        Q11 & 0.414 \\
        Q4 & 0.402 \\
        \hline
      \end{tabular}
    }
    \subfloat[PC3]{
      \begin{tabular}{c|c}
        \hline
        質問番号 & 主成分負荷量 \\ \hline \hline
        Q16 & 0.502 \\
        Q12 & 0.478 \\
        Q9 & -0.404 \\
        Q5 & 0.347 \\
        Q10 & -0.345 \\
        \hline
      \end{tabular}
    }
    \vspace{1cm}
    \subfloat[PC5]{
      \begin{tabular}{c|c}
        \hline
        質問番号 & 主成分負荷量 \\ \hline \hline
        Q7 & -0.366 \\
        Q15 & 0.328 \\
        Q11 & 0.319 \\
        Q17 & 0.299 \\
        Q9 & -0.295 \\
        \hline
      \end{tabular}
    }
    \subfloat[PC6]{
      \begin{tabular}{c|c}
        \hline
        質問番号 & 主成分負荷量 \\ \hline \hline
        Q8 & 0.568 \\
        Q5 & -0.438 \\
        Q4 & -0.353 \\
        Q6 & -0.338 \\
        Q15 & 0.194 \\
        \hline
      \end{tabular}
    }
    \subfloat[PC7]{
      \begin{tabular}{c|c}
        \hline
        質問番号 & 主成分負荷量 \\ \hline \hline
        Q18 & -0.609 \\
        Q2 & 0.250 \\
        Q5 & 0.221 \\
        Q15 & 0.211 \\
        Q19 & 0.207 \\
        \hline
      \end{tabular}
    }
  \end{center}
\end{table}

このデータを元に、各主成分に命名した一覧は以下の通りである。

\begin{description}
  \item[PC1] {\bf 好感度}
  \begin{itemize}
    \item 各要素に対する好感度を問うた質問が主成分負荷量絶対値の上位を独占したため
    \item 好感度を問うた質問同士の主成分負荷量絶対値の差があまりなかったため
  \end{itemize}

  \item[PC2] {\bf 非映像洗練性}
  \begin{itemize}
    \item 雰囲気・映像に対する洗練性を問うた質問が主成分負荷量絶対値の上位2つであったため
    \item その2問の主成分負荷量絶対値が以下の質問に対して差をあけていたため
    \item なお、PC2の重回帰式上の係数が負であるため、洗練されていないほど目的変数上昇に寄与することを意味する
  \end{itemize}

  \item[PC3] {\bf 聴きやすさ}
  \begin{itemize}
    \item 歌詞の分かりやすさ・メロディの聞き取りやすさが主成分負荷量絶対値の上位2つであったため
    \item その2問の主成分負荷量絶対値が以下の質問に対して差をあけていたため
    \item なお、PC3の重回帰式上の係数が負であるため、洗練されていないほど目的変数上昇に寄与することを意味する
  \end{itemize}

  \item[PC5] {\bf 楽曲洗練性}
  \begin{itemize}
    \item 楽曲の歌詞・メロディに対する洗練性を問うた質問が上位に入っていたため
    \item なお、歌詞・メロディのみならず、PC2で採用された映像に対する洗練性を問うた質問も上位に入っていた。
    \begin{itemize}
      \item しかしこの質問の主成分負荷量が負だったため、洗練されていないほど目的変数上昇に寄与することを意味する。
      \item 楽曲の歌詞・メロディに対する洗練性を問うた質問の主成分負荷量は正であった
    \end{itemize}
  \end{itemize}

  \item[PC6] {\bf 映像に対するアーティストの非親和性}
  \begin{itemize}
    \item アーティストの親しみやすさと映像の見やすさを問うた質問が上位2つを占めたため
    \begin{itemize}
      \item 主成分負荷量の正負では、Q8:映像の見やすさが正、Q5:アーティストの親しみやすさが負であった。
    \end{itemize}
    \item その2問の主成分負荷量絶対値が以下の質問に対して差をあけていたため
  \end{itemize}

  \item[PC7] {\bf 印象度}
  \begin{itemize}
    \item PVの印象度を問うた質問が、他の質問の主成分負荷量絶対値に対して大差をつけて最上位となっていたため
  \end{itemize}
\end{description}

\subsubsection{グループの類推}
調査結果に対して、クラスタ解析を行った結果が以下の図\ref{fig:clusterPV}の通りである。
\begin{figure}[h]
  \begin{center}
    \includegraphics[width=10cm]{clusterPV.bmp}
    \caption{PV16調査結果に対するクラスタ分析結果}
    \label{fig:clusterPV}
  \end{center}
\end{figure}
目的変数Q20に最も近い位置の質問がQ14となった。
そのQ14は、PC1における主成分負荷量絶対値が最大だった質問でもあった。
また、図右部分に各要素の洗練性を問うた質問が集中している他、
各主成分を構成していた質問が近い関係にある場合が多かった。
なお、PC7を構成していたPVの印象度を問うたQ18に最も近かったのは、
メロディの聴きやすさを問うたQ12であった。
更にPC3・PC6・PC7を構成する質問群は、
目的変数Q20の位置関係からかなり離れた場所に位置していたことも読み取れる。

\section{結論}
\subsection{仮説に対する検証}
第\ref{sc:intro}章で列挙した仮説について、
それぞれを解析結果と照らし合わせて検証した結果をまとめた。
\subsubsection{洗練性とPV好感度との関連}
\label{sc:pvsenren}
このPVにおける洗練性とPV好感度との関連だが、洗練されると好感度上昇に繋がる要素と
洗練されなくとも好感度上昇に繋がる要素との2つに分けられることが判明した。
第\ref{sc:pvpcq}節によると、目的変数$ PV_{like} $に洗練性が関連する主成分はPC2とPC5である。
その中で、PC2では雰囲気・映像が洗練されていない方が$ PV_{like} $に良い影響を与える傾向が示唆されている。
一方PC5では、歌詞・メロディといった楽曲を構成する要素については、
洗練されていた方が$ PV_{like} $に良い影響を与える傾向を示す結果が出た。

よって、映像・雰囲気作りにおいては、洗練されていなくとも好感度上昇に寄与する可能性が示唆された。
それと同時に、歌詞・メロディといった音楽的要素においては、洗練されていた方が好感度上昇に寄与する可能性も示唆された。

\subsubsection{迫力とPV好感度との関連}
各要素の迫力を問うた質問はQ9, 13であったが、
いずれも第\ref{sc:pvpcq}節で調べたPV16における主成分負荷量の上位には入っていなかった。
唯一PC3に関して言えば主成分負荷量絶対値の第三位に入っていたものの、
主成分負荷量絶対値は上2つと少々水をあけられている。
よって迫力とPV好感度との関連は、このPVに関して言えば大きくないことが推察される。

\subsubsection{メロディ・歌詞の聞き取りやすさとPV好感度との関連}
メロディの聞きやすさはQ12、歌詞の分かりやすさはQ16であった。
第\ref{sc:pvpcq}節により、PV16においてQ12・Q16は共にPC3の聴きやすさを構成する質問であった。
これはPV16のPV好感度への影響度においては無視できない質問と考えている。

\section{考察}
\subsection{PV全体調査結果とPV16との傾向の差異}
PV全体とPV16の違いとしてあるのが、主成分の数である。

表\ref{tb:allpcselect}がPV全体調査結果に対する主成分毎のデータを示しているが、
累積寄与率・固有値共にPC4以降は信頼性の薄いものとなった。
これに対して表\ref{tb:pv16pcs}がPV16の調査結果を見ると、
累積寄与率・固有値共に値の変化がゆるく、多くの主成分を採用せざるを得なくなった。

これは、PV16においてはあまり顕著な傾向が現れにくい作品であることを示唆している。


もう1つ大きな差異として捉えられるのは第\ref{sc:pvsenren}節で記述した洗練性である。

2者の共通点としては、一部の要素における洗練性が目的変数$PV_{like}$に影響することであるが、
映像・雰囲気の洗練性で差異が発生する。
PV全体の傾向としては第\ref{sc:allfaclor}節の通り、
映像・雰囲気の洗練度の主成分負荷量が正であることから、$PV_{like}$とは正の相関であることが確認できる。
しかしながらPV16においては、2者の主成分負荷量が負であったことから$PV_{like}$とは負の相関があることが確認できる。

これにより、PV16の製作者はあえて映像・雰囲気の洗練性を落とすことによって、
$PV_{like}$上昇へアプローチをかけた可能性が示唆されるのではないかと考えている。

\subsection{実験全体について}
今回の実験では、PVという3−6分の映像作品に対して多くの観点
(雰囲気・映像・アーティスト・メロディ・歌詞)で調査・解析を行うことから、
PV好感度に影響を与えやすい要素を求めることを行った。

この解析の意義には、個人的には作品によってどのような意図でPV好感度へアプローチをかけたのかを、
分析という観点から突き止めるところにあると考えている。

その一例として、先述のPV16における洗練性に対するアプローチのかけ方が挙げられる。
全体調査で洗練性が$PV_{like}$と強い関連がある点が示唆されていたが、
PV16では映像・雰囲気の洗練性が低いほうが$PV_{like}$に良い影響を与える点が示唆されていた。
これは、PVを制作するにあたってあえて逆を攻めることで印象に残す(印象深さはPC7に影響する)
というアプローチをかけたのではないかという推察につなげることが可能となっている。

逆にこの分析では及ばない点としては、解析結果があまり具体的な示唆を持てない点が挙げられる。
PV全体の解析結果・PV16解析結果共に、最も$PV_{like}$に影響を与えるのは各要素に対する好感度だった。
この好感度を上げるために必要なアプローチは、他の主成分に対して具体性を得難い要素があるため、
解析結果によってはあまり$PV_{like}$上昇に向けた具体性ある提案ができない状況が出来かねないのではないかと考えている。

\end{document}
