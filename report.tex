\documentclass[11pt,a4paper, uplatex]{jsarticle}
%
\usepackage{amsmath,amssymb}
\usepackage{bm}
\usepackage[dvipdfmx]{graphicx}
\usepackage{ascmac}
\usepackage{listings}
\usepackage{underscore}
\usepackage{subfig}
\lstset{
    frame=single,
    numbers=left,
    tabsize=2
}
%
\setlength{\textwidth}{\fullwidth}
\setlength{\textheight}{40\baselineskip}
\addtolength{\textheight}{\topskip}
\setlength{\voffset}{-0.2in}
\setlength{\topmargin}{0pt}
\setlength{\headheight}{0pt}
\setlength{\headsep}{0pt}
%
\newcommand{\divergence}{\mathrm{div}\,}  %ダイバージェンス
\newcommand{\grad}{\mathrm{grad}\,}  %グラディエント
\newcommand{\rot}{\mathrm{rot}\,}  %ローテーション
%
\title{メディア情報学実験・メディア分析 課題レポート}
\author{1510151  栁 裕太}
\date{\today}
\begin{document}
\maketitle
\section{序論・仮説}
今回の実験では、"きよしのズンドコ節"という曲を扱った(PV16)。
このPVでは昭和の歌謡曲のエッセンスを入れたことによって、
筆者のような20代前半にとっては個人的にはかなり印象深いPVであった。
よって今回は、以下の仮説を立ててから解析に臨むことにした。

\begin{itemize}
  \item PVを構成する任意の要素が洗練されていなくとも、好感度には影響しない
  \item 映像・メロディに迫力がなくとも、好感度には影響しない
  \item 聞き取りやすいメロディ・歌詞は好感度上昇に寄与する
\end{itemize}

\section{調査結果分析}
\subsection{a-sum分析(調査全体)}
\subsubsection{主成分抜粋}
\subsubsection{重回帰式による検証}
\subsubsection{目的関数に寄与する主成分の選定}
\subsubsection{重回帰式による再検証}
\subsubsection{主成分を構成する質問・主成分命名}
\subsubsection{グループの類推}

\subsection{PV16分析(きよしのズンドコ節)}
\subsubsection{主成分抜粋}
主成分抜粋においては、累積寄与率と固有値の2つのデータを基準に足切りを行った。
なお、PC11以降は省略している。

\begin{table}[htbp]
  \begin{center}
    \caption{主成分毎の累積寄与率と固有値}
    \begin{tabular}{c|c|c}
      \hline
      主成分番号 & 累積寄与率(\%) & 固有値 \\ \hline \hline
      PC1 & 38.38390791 & 7.29294250348674 \\
      PC2 & 48.55050945 & 1.93165429251404 \\
      PC3 & 55.79636735 & 1.37671299972187 \\
      PC4 & 61.89941496 & 1.15957904665 \\
      PC5 & 67.01570809 & 0.972095695622575 \\
      PC6 & 71.71468219 & 0.892805078337444 \\
      PC7 & 75.76539908 & 0.769636208492113 \\
      PC8 & 79.51202683 & 0.711859273699518 \\
      PC9 & 82.66178688 & 0.598454408935556 \\
      PC10 & 85.20054444 & 0.482363935634707 \\
      \hline
    \end{tabular}
  \end{center}
\end{table}

講義内では、以下の条件で足切りすることが推奨されていた。

\begin{itemize}
  \item 累積寄与率が80\%以下の主成分
  \item 固有値が1以上の主成分
\end{itemize}

前者であればPC8、後者であればPC4までとなるが、
両者のデータ共に値が著しく変化する境界があまり明瞭ではない。
そこで、前者の広い基準を採用し、解析後のP値等によって解析対象から外すことにした。

\subsubsection{重回帰式による検証}
PC1からPC8まで全ての主成分を対象に重回帰分析を行った。その結果は以下の通りである。

\begin{table}[htbp]
  \begin{center}
    \caption{重回帰分析結果: 各主成分の係数}
    \begin{tabular}{c|c|c|c|c|c|c}
      \hline
      主成分番号 & 偏回帰係数 & 標準誤差 & t値 & P値 & 標準化偏回帰係数 & トレランス \\ \hline \hline
      PC1 & -0.282 & 0.0119 & -23.6 & 2.09e-64 & -0.765 & 1 \\
      PC2 & -0.171 & 0.0232 & -7.38 & 2.66e-12 & -0.239 & 1 \\
      PC3 & -0.197 & 0.0275 & -7.15 & 1.05e-11 & -0.232 & 1 \\
      PC4 & -0.0489 & 0.0300 & -1.63 & 0.104 & -0.0529 & 1 \\
      PC5 & 0.143 & 0.0327 & 4.39 & 1.72e-05 & 0.142 & 1 \\
      PC6 & 0.105 & 0.0341 & 3.07 & 0.00242 & 0.0994 & 1 \\
      PC7 & -0.156 & 0.0368 & -4.24 & 3.26e-05 & -0.137 & 1 \\
      PC8 & -0.00994 & 0.0382 & -0.260 & 0.795 & -0.00843 & 1 \\
      定数項 & 3.15 & 0.0323 & 97.7 & 2.64e-195 & NA & NA \\
    \end{tabular}
    \label{one}
  \end{center}
\end{table}

\begin{table}[htbp]
  \begin{center}
    \caption{再重回帰分析結果: 重相関係数・自由度調整済重相関係数}
    \begin{tabular}{c|c|c}
      \hline
      重相関係数 & 重相関係数の2乗 & 自由度調整済重相関係数の2乗 \\ \hline \hline
      0.864590803140089 & 0.747517256874424 & 0.739101165436905
    \end{tabular}
    \label{Rs}
  \end{center}
\end{table}

\begin{table}[htbp]
  \begin{center}
    \caption{再重回帰分析結果: 分散分析}
    \begin{tabular}{c|c|c|c|c|c}
      \hline
      項目 & 平方和 & 自由度 & 平均平方 & F値 & P値 \\ \hline \hline
      回帰 & 184 & 8 & 23.0 & 88.8 & 2.36e-67 \\
      残差 & 62.2 & 240 & 0.259 & NA & NA \\
      全体 & 246 & 248 & 0.993 & NA & NA \\
    \end{tabular}
    \label{annova}
  \end{center}
\end{table}

これを重回帰式にすると、以下の通りとなった。なお、目的変数は$ PV_{like} $とした。

\begin{equation}
  \begin{split}
    PV_{like} &= -0.282PC_1 - 0.171PC_2 - 0.197PC_3 - 0.0489PC_4 \\
    &\quad + 0.143PC_5 + 0.105PC_6 - 0.156PC_7 - 0.00994PC_8 + 3.15
  \end{split}
\end{equation}

この目的変数$ PV_{like} $に対して$ PC_1 $から$ PC_8 $までを説明変数として重回帰分析を
行った結果、決定係数は$ R^2 = .75 $であり、0.1\%水準で有意であった。
($ F(8, 240) = 88.8 $)

\subsubsection{目的関数に寄与する主成分の選定}
\label{onesec}

主成分の選定において着目したのがP値の列である。講義内では$ p < 0.005 $が望ましいとされた。
今回の解析では、表\ref{one}のP値の列によると、
PC4, PC8がこの基準を満たしていないため、
除外して重回帰分析を再度行うことが適当であると判断した。

\subsubsection{重回帰式による再検証}
\ref{onesec}より、PC4, PC8を除外した状態でもう一度重回帰分析を行った。
その結果得られたデータは次の表\ref{two}の通りである。

\begin{table}[htbp]
  \begin{center}
    \caption{再重回帰分析結果: 各種成分の係数}
    \begin{tabular}{c|c|c|c|c|c|c}
      \hline
      主成分番号 & 偏回帰係数 & 標準誤差 & t値 & P値 & 標準化偏回帰係数 & トレランス \\ \hline \hline
      PC1 & -0.282 & 0.0120 & -23.5 & 1.64E-64 & -0.765 & 1 \\
      PC2 & -0.171 & 0.0232 & -7.36 & 2.79E-12 & -0.240 & 1 \\
      PC3 & -0.197 & 0.0275 & -7.14 & 1.10E-11 & -0.232 & 1 \\
      PC5 & 0.143 & 0.0328 & 4.38 & 1.77E-05 & 0.142 & 1 \\
      PC6 & 0.105 & 0.0342 & 3.06 & 0.00245 & 0.0994 & 1 \\
      PC7 & -0.156 & 0.0368 & -4.23 & 3.33E-05 & -0.137 & 1 \\
      定数項 & 3.15 & 0.0323 & 97.6 & 2.45E-196 & NA & NA \\
    \end{tabular}
    \label{two}
  \end{center}
\end{table}

\begin{table}[htbp]
  \begin{center}
    \caption{再重回帰分析結果: 重相関係数・自由度調整済重相関係数}
    \begin{tabular}{c|c|c}
      \hline
      重相関係数 & 重相関係数の2乗 & 自由度調整済重相関係数の2乗 \\ \hline \hline
      0.862928027094451 & 0.744644779945121 & 0.738313658786736
    \end{tabular}
    \label{Rs-new}
  \end{center}
\end{table}

\begin{table}[htbp]
  \begin{center}
    \caption{再重回帰分析結果: 分散分析}
    \begin{tabular}{c|c|c|c|c|c}
      \hline
      項目 & 平方和 & 自由度 & 平均平方 & F値 & P値 \\ \hline \hline
      回帰 & 183 & 6 & 30.6 & 118 & 7.69e-69 \\
      残差 & 62.9 & 242 & 0.260 & NA & NA \\
      全体 & 246 & 248 & 0.993 & NA & NA \\
    \end{tabular}
    \label{annova-new}
  \end{center}
\end{table}

これを重回帰式にすると、以下の式\ref{formula:two}の通りとなった。

\begin{equation}
  \label{formula:two}
  \begin{split}
    PV_{like} &= -0.282PC_1 - 0.171PC_2 - 0.197PC_3 - \\
    &\quad + 0.143PC_5 + 0.105PC_6 - 0.156PC_7 + 3.15
  \end{split}
\end{equation}

この分析結果をまとめると、目的変数$ PV_{like} $に対して
$ PC_1, PC_2, PC_3, PC_5, PC_6, PC_7$
までを説明変数として重回帰分析を行った結果、
決定係数は$ R^2 = .74 $であり、0.1\%水準で有意であった。
($ F(6, 242) = 118 $)

\subsubsection{主成分を構成する質問・主成分命名}
PV16の主成分を構成する質問の選出には、各質問に対応した主成分負荷量のみを対象とした。
これは、主成分得点は全ての生データそのものに対応しており、質問を選出するのが困難だったためである。

以下の表が、各主成分に対して主成分負荷量の絶対値が最も大きかった5つの質問とその負荷量を表している。

\begin{table}[htbp]
  \begin{center}
    \caption{各主成分に対する主成分負荷量絶対値TOP5}
    \subfloat[PC1]{
      \begin{tabular}{c|c}
        \hline
        質問番号 & 主成分負荷量 \\ \hline \hline
        Q14 & 0.801 \\
        Q17 & 0.756 \\
        Q3 & 0.727 \\
        Q10 & 0.724  \\
        Q6 & 0.721 \\
        \hline
      \end{tabular}
    }
    \subfloat[PC2]{
      \begin{tabular}{c|c}
        \hline
        質問番号 & 主成分負荷量 \\ \hline \hline
        Q1 & 0.604 \\
        Q7 & 0.548 \\
        Q2 & 0.418 \\
        Q11 & 0.414 \\
        Q4 & 0.402 \\
        \hline
      \end{tabular}
    }
    \subfloat[PC3]{
      \begin{tabular}{c|c}
        \hline
        質問番号 & 主成分負荷量 \\ \hline \hline
        Q16 & 0.502 \\
        Q12 & 0.478 \\
        Q9 & 0.404 \\
        Q5 & 0.347 \\
        Q10 & 0.345 \\
        \hline
      \end{tabular}
    }
    \vspace{1cm}
    \subfloat[PC5]{
      \begin{tabular}{c|c}
        \hline
        質問番号 & 主成分負荷量 \\ \hline \hline
        Q7 & 0.366 \\
        Q15 & 0.328 \\
        Q11 & 0.319 \\
        Q17 & 0.299 \\
        Q9 & 0.295 \\
        \hline
      \end{tabular}
    }
    \subfloat[PC6]{
      \begin{tabular}{c|c}
        \hline
        質問番号 & 主成分負荷量 \\ \hline \hline
        Q8 & 0.568 \\
        Q5 & 0.438 \\
        Q4 & 0.353 \\
        Q6 & 0.338 \\
        Q15 & 0.194 \\
        \hline
      \end{tabular}
    }
    \subfloat[PC7]{
      \begin{tabular}{c|c}
        \hline
        質問番号 & 主成分負荷量 \\ \hline \hline
        Q18 & 0.609 \\
        Q5 & 0.250 \\
        Q4 & 0.221 \\
        Q6 & 0.211 \\
        Q15 & 0.207 \\
        \hline
      \end{tabular}
    }
  \end{center}
\end{table}

このデータを元に、各主成分に命名した一覧は以下の通りである。

\begin{description}
  \item[PC1] {\bf 好感度}
  \begin{itemize}
    \item 各要素に対する好感度を問うた質問が主成分負荷量絶対値の上位を独占したため
    \item 好感度を問うた質問同士の主成分負荷量絶対値の差があまりなかったため
  \end{itemize}

  \item[PC2] {\bf 映像洗練性}
  \begin{itemize}
    \item 雰囲気・映像に対する洗練性を問うた質問が主成分負荷量絶対値の上位2つであったため
    \item その2問の主成分負荷量絶対値が以下の質問に対して差をあけていたため
  \end{itemize}

  \item[PC3] {\bf 聴きやすさ}
  \begin{itemize}
    \item 歌詞の分かりやすさ・メロディの聞き取りやすさが主成分負荷量絶対値の上位2つであったため
    \item その2問の主成分負荷量絶対値が以下の質問に対して差をあけていたため
  \end{itemize}

  \item[PC5] {\bf 楽曲洗練性}
  \begin{itemize}
    \item 楽曲の歌詞・メロディに対する洗練性を問うた質問が上位に入っていたため
    \item なお、歌詞・メロディのみならず、PC2で採用された映像に対する洗練性を問うた質問も上位に入っていた
  \end{itemize}

  \item[PC6] {\bf 映像に対するアーティストの親和性}
  \begin{itemize}
    \item アーティストの親和性と映像の見やすさを問うた質問が上位2つを占めたため
    \item その2問の主成分負荷量絶対値が以下の質問に対して差をあけていたため
  \end{itemize}

  \item[PC7] {\bf 印象度}
  \begin{itemize}
    \item PVの印象度を問うた質問が、他の質問の主成分負荷量絶対値に対して大差をつけて最上位となっていたため
  \end{itemize}
\end{description}

\subsubsection{グループの類推}
調査結果に対して、クラスタ解析を行った結果が以下の図\ref{fig:clusterPV}の通りである。
\begin{figure}[h]
  \begin{center}
    \includegraphics[width=10cm]{clusterPV.bmp}
    \caption{PV16調査結果に対するクラスタ分析結果}
    \label{fig:clusterPV}
  \end{center}
\end{figure}
目的変数Q20に最も近い位置の質問がQ14となった。
そのQ14は、PC1における主成分負荷量絶対値が最大だった質問でもあった。
また、図左部分に各要素の洗練性を問うた質問が集中している他、
各主成分を構成していた質問が近い関係にある場合が多かった。
なお、PC7を構成していたPVの印象度を問うたQ18に最も近かったのは、
メロディの聴きやすさを問うたQ12であった。

\section{結論}
\section{考察}



\end{document}
