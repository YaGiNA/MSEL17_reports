\documentclass[11pt,a4paper, uplatex]{jsarticle}
%
\usepackage{amsmath,amssymb}
\usepackage{bm}
\usepackage[dvipdfmx]{graphicx}
\usepackage{ascmac}
\usepackage{listings}
\usepackage{underscore}
\lstset{
    frame=single,
    numbers=left,
    tabsize=2
}
%
\setlength{\textwidth}{\fullwidth}
\setlength{\textheight}{40\baselineskip}
\addtolength{\textheight}{\topskip}
\setlength{\voffset}{-0.2in}
\setlength{\topmargin}{0pt}
\setlength{\headheight}{0pt}
\setlength{\headsep}{0pt}
%
\newcommand{\divergence}{\mathrm{div}\,}  %ダイバージェンス
\newcommand{\grad}{\mathrm{grad}\,}  %グラディエント
\newcommand{\rot}{\mathrm{rot}\,}  %ローテーション
%
\title{メディア情報学実験・メディア分析 課題レポート}
\author{1510151  栁 裕太}
\date{\today}
\begin{document}
\maketitle
\section{序論・仮説}
今回の実験では、"きよしのズンドコ節"という曲を扱った(PV16)。
このPVでは昭和の歌謡曲のエッセンスを入れたことによって、
筆者のような20代前半にとっては個人的にはかなり印象深いPVであった。
よって今回は、以下の仮説を立ててから解析に臨むことにした。

\begin{itemize}
  \item PVを構成する任意の要素が洗練されていなくとも、好感度には影響しない
  \item 映像・メロディに迫力がなくとも、好感度には影響しない
  \item 聞き取りやすいメロディ・歌詞は好感度上昇に寄与する
\end{itemize}

\section{調査結果分析}
\subsection{主成分抜粋}
\subsection{目的関数に寄与する主成分の選定・再解析}
\subsection{主成分を構成する質問・主成分命名}
\subsection{グループの類推}


\section{結論}
\section{考察}



\end{document}
